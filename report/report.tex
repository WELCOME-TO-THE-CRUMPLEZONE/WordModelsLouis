%\setlength{\parindent}{0pt}
\documentclass{article}
\usepackage{amsmath}
\usepackage{amssymb}
\usepackage{amsthm}
\usepackage{latexsym}

\usepackage{amsopn}
\DeclareMathOperator{\stab}{Stab}
\DeclareMathOperator{\perm}{Perm}
\DeclareMathOperator{\im}{im}
\DeclareMathOperator{\Aut}{Aut}
\DeclareMathOperator{\Hom}{Hom}
\DeclareMathOperator{\Perm}{Perm}
\DeclareMathOperator{\Frac}{Frac}
\DeclareMathOperator{\Pic}{Pic}
\DeclareMathOperator{\id}{id}
\DeclareMathOperator{\Tr}{Tr}
\DeclareMathOperator{\Spec}{Spec}
\DeclareMathOperator{\Proj}{Proj}
\DeclareMathOperator{\codim}{codim}


\usepackage[dvipsnames]{xcolor}
 
\definecolor{mypink1}{rgb}{0.858, 0.188, 0.478}

%\usepackage{color}
%\definecolor{keywordcolor}{rgb}{0.7, 0.1, 0.1}   % red
%\definecolor{commentcolor}{rgb}{0.4, 0.4, 0.4}   % grey
%\definecolor{symbolcolor}{rgb}{0.0, 0.1, 0.6}    % blue
%\definecolor{sortcolor}{rgb}{0.1, 0.5, 0.1}      % green
%\usepackage{listings}
%\def\lstlanguagefiles{lstlean.tex} 
%\lstset{language=lean}

\usepackage{enumitem}
\usepackage{tikz-cd}

\usepackage{tikz,tkz-euclide}
\usetikzlibrary{arrows,calc,intersections}
%\usetkzobj{all}

\usepackage{enumitem}
\usepackage[margin=2.2cm]{geometry}

\newcommand{\ideal}{\ensuremath{\triangleleft}}
\newcommand{\ol}{\ensuremath{\overline}}
\newcommand{\p}{\ensuremath{\mathfrak{p}}}
\newcommand{\m}{\ensuremath{\mathfrak{m}}}
\newcommand{\A}{\ensuremath{\mathbb{A}}}
\newcommand{\Z}{\ensuremath{\mathbb{Z}}}
\newcommand{\C}{\ensuremath{\mathbb{C}}}
\newcommand{\R}{\ensuremath{\mathbb{R}}}
\newcommand{\Q}{\ensuremath{\mathbb{Q}}}
\newcommand{\N}{\ensuremath{\mathbb{N}}}
\newcommand{\F}{\ensuremath{\mathbb{F}}}
\renewcommand{\P}{\ensuremath{\mathbb{P}}}
\newcommand{\Ox}{\mathscr{O}}

\newcommand{\q}{\ensuremath{\mathfrak{q}}}
%\newcommand{\N}{\ensuremath{\mathbb{N}}}

\usepackage{mathrsfs}
%\usepackage{fontspec}
%\usepackage{mathtools}
%\usepackage{unicode-math}

\usepackage{natbib}
\bibliographystyle{humannat}
%\bibliographystyle{unsrtnat}
%\bibliographystyle{abbrvnat}

\theoremstyle{definition}
\newcounter{dummy} \numberwithin{dummy}{section}
\newtheorem{lemma}[dummy]{Lemma}
%\newtheorem*{lemma*}[dummy]{Lemma}
\newtheorem{prop}[dummy]{Proposition}
\newtheorem{defi}[dummy]{Definition}
\newtheorem{cor}[dummy]{Corollary}
\newtheorem{example}[dummy]{Example}
\newtheorem{thm}[dummy]{Theorem}



\newcommand*{\DashedArrow}[1][]{\mathbin{\tikz [baseline=-0.25ex,-latex, dashed,#1] \draw [#1] (0pt,0.5ex) -- (1.3em,0.5ex);}}%
\newcommand{\dto}{\DashedArrow[->,densely dashed]}
%\newcommand{\da}{\DashedArrow}

%\usepackage{microtype}
\usepackage[activate={true,nocompatibility},final,tracking=true,kerning=true,spacing=true,factor=1100,stretch=10,shrink=10]{microtype}
\author{Louis Carlin -- u6384109}
\title{World Models with MDRNNs}
\usepackage[pdftex]{hyperref}
\hypersetup{
  colorlinks=true
}
\begin{document}
\maketitle

\section{Introduction}
%background stuff of other world models
When humans make decisions we often use an internal model of the problem at hand to inform our decision.
This allows us to predict or calculate how the world might unfold based on our actions.
For example, a chess player doesn't make a moved based solely on the current sate of the board, but instead uses their knowledge of the rules to look ahead and anticipate potential consequences of each move and evaluate them accordingly.
Classical game AIs achieved great success on the same principle: a classical chess AI uses its knowledge of legal moves to heuristically search the tree of possible moves, picking the best continuation according to the MiniMax algorithm.

In reinforcement learning we seek to develop AI which are capable of \textit{learning} to play games.
Reinforcement learning agents equipped with models of their environment typically outperform similar agents with no model.
Constructing a model of a complex environment by hand can be time consuming or infeasible, thus one area of active research in reinforcement learning is developing agents which are capable of learning their own internal models of the world without outside help.
This is not a new area of research... %TODO





%overview of model V-M-C



\section{Variational Autoencoders}
%autoencoders (show old example?)
%show new example


\section{MDRNNs}

\section{Results?}

\section{The gym environment}

\clearpage
\bibliography{ref} 
\end{document}
